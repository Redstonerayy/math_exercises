\documentclass[10pt,a4paper]{article}

\usepackage[margin={2in, 0.5in}]{geometry}
\usepackage[]{graphicx,amsmath,amsfonts,array}

% Remove Indentation
\setlength{\parindent}{0cm}

\begin{document}
	
\title{Übungsaufgaben zu Ableitungsregeln, AB1}

%redefine maketitle  to make the title it bigger
\makeatletter
\def\@maketitle{%
  \newpage
  \null
  \vskip 2em%
  \begin{center}%
  \let \footnote \thanks
    {\Huge\bfseries\@title \par}%
    \vskip 1.5em%
    {\large
      \lineskip .5em%
      \begin{tabular}[t]{c}%
        \@author
      \end{tabular}\par}%
    \vskip 1em%
    {\large \@date}%
  \end{center}%
  \par
  \vskip 1.5em}
\makeatother

\author{}
\date{}

\maketitle

\section*{Ableitungsregel: Konstantenregel}
Die Ableitung einer Konstanten ist immer Null, da sie irrelevant für die Steigung
der Funktion ist und nur wichtig, wenn es um den Entgültigen y-Wert geht.

Somit ist $f'(x) = 0$ die 1. Ableitung von $f(x) = 5$, da jede Konstante null wird.
Dies gilt für jede Konstante $C$ wo $C  \in \mathbb{R}$, also wo $C$ eine reelle Zahl ist.

\section*{Ableitungsregel: Ableitung von x}
Die Ableitung von $x$ is $1$. Dies lässt sich so begründen, dass $f(x) = x = 1x$
eine Lineare Funktion mit der Steigung $m = 1$ ist, was dazu führt, dass die Steigung dieser
Funktion $1$ ist und somit auch $x$ in der Ableitungsfunktion mit $1$ dargestellt werden muss.

\section*{Aufgaben zu Ableitungsregeln}
Bilden sie die 1. Ableitung der Funktionen. \newline
Setzt Wissen zur Ableitung von Konstanten und von x voraus. \newline

\begin{tabular}{l l l}

	a) $f(x) = 5$        & b) $f(x) = x$        & c) $f(x) = 2x + 4$ \\
	d) $f(x) = 3x + 0$   & e) $f(x) = 1x + 5$   & f) $f(x) = 20x + 100$ \\
	g) $f(x) = 2*2x + 2$ & h) $f(x) = 0.5x + 0$ & i) $f(x) = 0x + 5$

\end{tabular}

\end{document}