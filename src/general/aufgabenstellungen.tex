\documentclass[10pt,a4paper]{article}

\usepackage{graphicx,geometry,amsmath,array}

\begin{document}

\title{Aufgabenstellungen zu Polynomfunktionen und anderen Funktionen}

%redefine maketitle  to make the title it bigger
\makeatletter
\def\@maketitle{%
  \newpage
  \null
  \vskip 2em%
  \begin{center}%
  \let \footnote \thanks
    {\Huge\bfseries\@title \par}%
    \vskip 1.5em%
    {\large
      \lineskip .5em%
      \begin{tabular}[t]{c}%
        \@author
      \end{tabular}\par}%
    \vskip 1em%
    {\large \@date}%
  \end{center}%
  \par
  \vskip 1.5em}
\makeatother

\author{}
\date{}

\maketitle

%docment contents
\section*{Untersuchen von Funktionen}

\begin{itemize}
    \item Ableitungen
    \item Nullstellen(Vielfachheit)
    \item y-Achsenabschnitt
    \item Extrempunkte
    \item Wendepunkte(Sattelpunkte)
    \item Symmetrie
    \item Funktionsgraph zeichnen
    \item Verhalten im Unendlichen
    \item Definitionsbereich
    \item Wertebereich
    \item Monotonie
    \item Krümmung
    \item Tangentengleichung
    \item Normalengleichung
\end{itemize}

\section*{Methoden/Kompetenzen}

\begin{itemize}
    \item Ableitungsregeln
    \item Vorgehen Kurvendiskussion
    \item Ableitungen mit Taschenrechner
    \item Intervalschreibweise
\end{itemize}

\end{document}