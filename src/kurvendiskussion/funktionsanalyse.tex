\documentclass[10pt,a4paper]{article}

\usepackage[margin={2in, 0.5in}]{geometry}
\usepackage[]{graphicx,amsmath,array,relsize}

\begin{document}

% Remove Indentation
\setlength{\parindent}{0cm}

\title{Analyse von Funktionen}

%redefine maketitle  to make the title it bigger
\makeatletter
\def\@maketitle{%
  \newpage
  \null
  \vskip 2em%
  \begin{center}%
  \let \footnote \thanks
    {\Huge\bfseries\@title \par}%
    \vskip 1.5em%
    {\large
      \lineskip .5em%
      \begin{tabular}[t]{c}%
        \@author
      \end{tabular}\par}%
    \vskip 1em%
    {\large \@date}%
  \end{center}%
  \par
  \vskip 1.5em}
\makeatother

\author{}
\date{}

\maketitle

\section*{Generell}

Eine Funktion kann vielfältig analysiert werden. Vorraussetzung sind dabei die Ableitungen
bis einschließlich der 3. Ableitung.

\section*{Nullstellen}
Die Nullstellen der Ursprungsfunktion können in mehreren Verfahren oder Kombination dieser bestimmt werden. Diese sind
\begin{itemize}
	\item Umformung(z. B. Ausklammern)
	\item pq-Formel
	\item Taschenrechner(solve)
	\item Polynomdivision(nicht behandelt)
\end{itemize}

Da die Umformung schwer einzugrenzen ist und die Lösung mit dem Taschenrechner einfach, wird nur die pq-Formel betrachtet.

Die pq-Formel lässt sich nur auf Quadratische Funktionen bzw. Funktionen des 2. Grades anwenden.
Dabei muss der Vorfaktor $a$ aus der Normalform $ax^2 + bx + c$ 1 sein.
$b$ wird im Folgenden als $p$ bezeichnet, $c$ als $q$.

Die pq-Formel lautet: \newline

$\mathlarger{x_{1,2} = -\frac{p}{2} \pm \sqrt[]{(\frac{p}{2})^2 - q}}$ \newline

Das Einsetzen ergibt in die Formel ergibt dann die Nullstellen der Funktion.

Wenn man von der Funktion $f(x) = x^2 + 4x + 2$ ausgeht, dann ist die zugehörige pq-Formel \newline

$\mathlarger{x_{1,2} = -\frac{4}{2} \pm \sqrt[]{(\frac{4}{2})^2 - 2}}$ \newline

Diese lässt sich vereinfachen zu \newline

$\mathlarger{x_{1,2} = -2 \pm \sqrt[]{4 - 2}}$

Ist allerdings die Summe unter der Wurzel negativ, so gibt es keine Lösungen.

\section*{y-Achsenabschnitt}

Der y-Achsenabschnitt ist der Wert der Funktion an der Stelle $x = 0$.
Er bezeichnet denn Schnittpunkt des Graphen mit der y-Achse und lässt sich
durch Einsetzen von Null in die Ursprungsfunktion berechnen. 

\section*{Extrempunkte}

\section*{Wendepunkte}

\section*{Symmetrie}

\section*{Funktionsgraph}

\section*{Verhalten im Unendlichen}

\section*{Definitionsbereich}

\section*{Wertebereich}

\section*{Monotonie}

\section*{Monotonie}

\section*{Krümmung}

\section*{Tangengleichung}

\section*{Normalengleichung}

\end{document}