\documentclass[10pt,a4paper]{article}

\usepackage[margin={1.5in, 0.5in}]{geometry}
\usepackage[]{graphicx,amsmath,amsfonts,array}

% Remove Indentation
\setlength{\parindent}{0cm}

\begin{document}
	
\title{Übungsaufgaben zu Extremstellen und Wendepunkten (1)}

%redefine maketitle  to make the title it bigger
\makeatletter
\def\@maketitle{%
  \newpage
  \null
  \vskip 2em%
  \begin{center}%
	\let \footnote \thanks
    {\Huge\bfseries\@title \par}%
    \vskip 1.5em%
    {\large
	\lineskip .5em%
	\begin{tabular}[t]{c}%
        \@author
      \end{tabular}\par}%
    \vskip 1em%
    {\large \@date}%
  \end{center}%
  \par
  \vskip 1.5em}
\makeatother

\author{}
\date{}

\maketitle

\section*{Bestimmen von Extremstellen}
Die Extremstellen eines Graphen sind die höchsten oder niedrigsten Punkte des Graphen.
Dabei unterscheided man zwischen lokalen und globalen Minima und Maxima. Global bedeutet,
dass es auf den ganzen Graphen der höchste oder tiefste Punkt ist, lokal, das es nur in einem Bereich
der höchste oder tiefste Punkt ist.
\newline

Die Extremstellen einer Funktion $f(x)$ sind an den Stellen, wo $f'(x) = 0$ gilt, also wo
die Ursprungsfunktion keine Steigung hat. Außerdem gibt die 2. Ableitung, wenn man sie mit Null
gleichgesetzt, also $f''(x) = 0$, aufschluss darüber, ob es ein Hoch- oder Tiefpunkt ist.
Ist der Wert der 2. Ableitung an der Stelle x größer als $0$, so liegt ein Tiefpunkt vor, ist der Wert kleiner
als $0$, dann ist es ein Hochpunkt.
\newline

Somit lässt sich festhalten, dass die zur Berechnung der Extremstellen die 1. Ableitung der Funktion
mit $0$ gleichgesetzt werden muss. An den Stellen wo nun $f'(x) = 0$ ist, muss nun x in die 2. Ableitung
eingesetzt werden, um zu prüfen, ob es ein Hoch- oder Tiefpunkt ist.
Zum Schluss können noch durch Einsetzen in die Ursprungsfunktion die tatsächlichen Punkte
berechnet werden.
\newline

1. Ableitung gleich Null setzen: $f'(x) = 0$

In 2. Ableitung einsetzen: $f''(x) > 0$ oder $f''(x) < 0$

Punkte berechen: $E( x | f(x) )$

\section*{Bestimmen von Wendepunkten}
Die Wendepunkte bei einem Graphen sind die Punkte, wo der Graph sein Krümmungsverhalten ändert.
Voraussetzung dafür ist, dass die 2. Ableitung an diesem Punkt gleich $0$ ist und die 3. Ableitung ungleich $0$.
\newline

Für einen Wendepunkt gilt also:

$f''(x) = 0$
$f'''(x) \neq 0$
\newline

Zur Bestimmung der Wendestellen müssen nun zunächst die Ableitungen gebildet werden und
die 2. Ableitung wird mit 0 gleichgesetzt. Anschließend überprüft man dann mit der 3. Ableitung,
ob es ein Wendepunkt ist.

\section*{Aufgaben zu Extremstellen und Wendepunkten}

\begin{tabular}{l l l}

	a) $f(x) = x^3 -3x + 2$        & b) $f(x) = x^3 + 3x + 1$  & c) $f(x) = 2x^3 + 1x - 2$ \\
	d) $f(x) = \frac{1}{3}x^3 + 5$ & e) $f(x) = 4x^2 + 3x + 1$ & f) $f(x) = 0.5x^3 + 1.5x + 1.5$
\end{tabular}

\end{document}