\documentclass[10pt,a4paper]{article}

\usepackage[margin={1.5in, 0.5in}]{geometry}
\usepackage[]{graphicx,amsmath,amsfonts,array}

% Remove Indentation
\setlength{\parindent}{0cm}

\begin{document}
	
\title{Lösungen zu Übungsaufgaben zu Extremstellen und Wendepunkten (1)}

%redefine maketitle  to make the title it bigger
\makeatletter
\def\@maketitle{%
  \newpage
  \null
  \vskip 2em%
  \begin{center}%
	\let \footnote \thanks
    {\Huge\bfseries\@title \par}%
    \vskip 1.5em%
    {\large
	\lineskip .5em%
	\begin{tabular}[t]{c}%
        \@author
      \end{tabular}\par}%
    \vskip 1em%
    {\large \@date}%
  \end{center}%
  \par
  \vskip 1.5em}
\makeatother

\author{}
\date{}

\maketitle

\section*{Lösungen zu Aufgaben zu Extremstellen und Wendepunkten}

\begin{tabular}{l l l}

	a) $f(x) = x^3 -3x + 2$        & b) $f(x) = x^3 + 3x + 1$  & c) $f(x) = 2x^3 + 1x - 2$ \\
	d) $f(x) = \frac{1}{3}x^3 + 5$ & e) $f(x) = 4x^2 + 3x + 1$ & f) $f(x) = 0.5x^3 + 1.5x + 1.5$
\end{tabular}

\end{document}