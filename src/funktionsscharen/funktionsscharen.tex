\documentclass[10pt,a4paper]{article}

\usepackage[margin={2in, 0.5in}]{geometry}
\usepackage[]{graphicx,amsmath,array}

\begin{document}

% Remove Indentation
\setlength{\parindent}{0cm}

\title{Funktionsscharen}

%redefine maketitle  to make the title it bigger
\makeatletter
\def\@maketitle{%
  \newpage
  \null
  \vskip 2em%
  \begin{center}%
  \let \footnote \thanks
    {\Huge\bfseries\@title \par}%
    \vskip 1.5em%
    {\large
      \lineskip .5em%
      \begin{tabular}[t]{c}%
        \@author
      \end{tabular}\par}%
    \vskip 1em%
    {\large \@date}%
  \end{center}%
  \par
  \vskip 1.5em}
\makeatother

\author{}
\date{}

\maketitle

\section*{Funktionsscharen}

Ein Funktionsschar ist eine Menge von Funktionen, die aus einer Funktion durch veränderung
eines Parameters geschieht, dem Scharparameter, auch k genannt.

Es lässt sich nun allgemein Untersuchen, wo die signifikaten Stellen des Graphen in Abhängigkeit
von k liegen.

Wenn man nun Beispielhaft die Funktion $f(x) = k \cdot x^3 - 2x^2$ nimmt mit dem Scharparameter k.

$f_{k}(x) = k \cdot x^3 - 2x^2$

$f'_{k}(x) = k \cdot 3x^2 - 4x^1$

$f''_{k}(x) = k \cdot 6^1 - 4$

Setzt man nun die 1. Ableitung mit 0 gleich, so erhält man die Nullstellen $x = 0$ und $x = \frac{4}{3k}$.
Die 2. Nullstelle lässt sich nun nach $k$ umformen, sodass man $k = \frac{4 - x}{3}$ erhält.

Um die Extrempunkte zu erhalten setzt man nun k in die Funktion für den x-Wert ein
und anschließend in die Ursprungsfunktion.

$EP(\frac{4}{3k}|f_{k}(\frac{4 - \frac{4}{3k}}{3}))$

\end{document}
