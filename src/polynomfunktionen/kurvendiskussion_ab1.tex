\documentclass[10pt,a4paper]{article}

\usepackage[margin={2in, 0.5in}]{geometry}
\usepackage[]{graphicx,amsmath,array}

\begin{document}

\title{Liste an Polynomfunktionen}

%redefine maketitle  to make the title it bigger
\makeatletter
\def\@maketitle{%
  \newpage
  \null
  \vskip 2em%
  \begin{center}%
  \let \footnote \thanks
    {\Huge\bfseries\@title \par}%
    \vskip 1.5em%
    {\large
      \lineskip .5em%
      \begin{tabular}[t]{c}%
        \@author
      \end{tabular}\par}%
    \vskip 1em%
    {\large \@date}%
  \end{center}%
  \par
  \vskip 1.5em}
\makeatother

\author{}
\date{}

\maketitle

\section*{Number x}

Bestimme die Nullstellen der Funktionen. \newline
Maximiere anschließend die 1. Ableitung und finde so mögliche Hochpunkte. \newline
Bestimme anschließend die Wendepunkte, wenn vorhanden. \newline
Stelle zuletzt Überlegungen zur Symetrie an und skizziere den Funktionsgraph. \newline

\begin{tabular}{l l l}

    1) \ f(x) = -3,5x + 14,7 & 2) \ f(x) = 2x - y - 5 & 3) \ f(x) = -3 \\
    4) \ f(x) = $x \cdot (x - 4,8)$ & 5) \ f(x) = $2x^2 - 18$ & 6) \ f(x) = $3x^2 + 6x$

\end{tabular}

\section*{Lösungen x}

Lösungen zu den Aufgaben. Nullstellen sind: \newline

1)

2)

3)

4)

5)

6)

\end{document}
