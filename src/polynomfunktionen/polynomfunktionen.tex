\documentclass[10pt,a4paper]{article}

\usepackage{graphicx,geometry,amsmath,array}

\begin{document}

\title{Polynomfunktionen}

%redefine maketitle  to make the title it bigger
\makeatletter
\def\@maketitle{%
  \newpage
  \null
  \vskip 2em%
  \begin{center}%
  \let \footnote \thanks
    {\Huge\bfseries\@title \par}%
    \vskip 1.5em%
    {\large
      \lineskip .5em%
      \begin{tabular}[t]{c}%
        \@author
      \end{tabular}\par}%
    \vskip 1em%
    {\large \@date}%
  \end{center}%
  \par
  \vskip 1.5em}
\makeatother

\author{}
\date{}

\maketitle

\section*{Number x}

Bestimme die Nullstellen der Funktionen 1-6. \newline

\begin{tabular}{l l l}

    1) \ f(x) = -3,5x + 14,7 & 2) \ f(x) = 2x - y - 5 & 3) \ f(x) = -3 \\
    4) \ f(x) = $x \cdot (x - 4,8)$ & 5) \ f(x) = $2x^2 - 18$ & 6) \ f(x) = $3x^2 + 6x$

\end{tabular}

\section*{Lösungen x}

Lösungen zu den Aufgaben. Nullstellen sind: \newline

1)

2)

3)

4)


\end{document}
