\documentclass[10pt,a4paper]{article}

\begin{document}

% Remove Indentation
\setlength{\parindent}{0cm}

\title{Potenzgesetze}

%redefine maketitle  to make the title it bigger
\makeatletter
\def\@maketitle{%
  \newpage
  \null
  \vskip 2em%
  \begin{center}%
  \let \footnote \thanks
    {\Huge\bfseries\@title \par}%
    \vskip 1.5em%
    {\large
      \lineskip .5em%
      \begin{tabular}[t]{c}%
        \@author
      \end{tabular}\par}%
    \vskip 1em%
    {\large \@date}%
  \end{center}%
  \par
  \vskip 1.5em}
\makeatother

\author{}
\date{}

\maketitle

\section*{Was ist eine Potenz}
Die Potenz $x^n$ beschreibt, dass die Zahl $x$ $n-1$ mal mit sich 
selbst multipliziert wird. Für $x^3$ ist also auch diese Schreibweise möglich:

$x \cdot x \cdot x$

Die Potenzgesetze beschreiben dabei, wie sich mit Potenzen einfacher rechnen lässt,
und wie Terme zusammengefasst und vereinfacht werden können zur Umformung.


\end{document}